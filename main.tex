\documentclass[uplatex,a4paper,11pt]{jsarticle} % uplatexオプションを入れる
%\documentclass{jarticle}
\usepackage{geometry}
%\geometry{left=25mm, right=25mm, top=30mm, bottom=30mm, truedimen}
\geometry{left=18.75mm, right=18.75mm, top=22.5mm, bottom=22.5mm, truedimen}

\usepackage[T1]{fontenc}             % T1エンコーディングにしてみる
\usepackage{plext}                   % pLaTeX付属の縦書き支援パッケージ
\usepackage{okumacro}                % jsclassesに同梱のパッケージ

\usepackage{titlesec}
\usepackage{array}
\usepackage{amssymb} 
\usepackage{amsmath} 
\usepackage{booktabs}
\usepackage{multirow}
\usepackage{midpage}


%\usepackage{latexsym}
\usepackage{mathtools}

% 画像系パッケージの読み込み
\usepackage{wrapfig}
\usepackage{caption}
\usepackage{subcaption}
\captionsetup{labelsep=quad}

%\usepackage{lscape}
\usepackage[dvipdfmx]{pdflscape}
\usepackage{comment}
\usepackage{url}

% pdfのしおり
\usepackage[dvipdfmx]{hyperref}         %ハイパーリンク
\usepackage{pxjahyper}                  %ハイパーリンク


\def\thefootnote{\arabic{footnote}}

\usepackage[version=3]{mhchem}      % 化学式用パッケージ(\ce{C + O2 -> CO2})

\setcounter{secnumdepth}{3}
\titlelabel{\thetitle.\:}
\titleformat{\section}{\normalfont\bfseries}{\thesection.}{1em}{}
\titleformat{\subsection}{\normalfont\bfseries}{\thesubsection}{1em}{}
\titleformat{\subsubsection}{\normalfont\bfseries}{\thesubsubsection}{1em}{}


% 番号フォーマット
\numberwithin{equation}{section}     % 数式番号フォーマット
\renewcommand{\thefigure}{\arabic{section}.\arabic{figure}}
\renewcommand{\thetable}{\arabic{section}.\arabic{table}}
\renewcommand{\labelenumi}{(\arabic{enumi})}
\makeatletter
\@addtoreset{equation}{section}
\@addtoreset{figure}{section}
\@addtoreset{table}{section}
\renewcommand{\@cite}[1]{\textsuperscript{(#1)}}     % 本文中の引用のフォーマット変更
\renewcommand{\@biblabel}[1]{(#1)}                   % 参考文献のフォーマット
\makeatother

% Fig. Tableに変更
% \renewcommand{\figurename}{Fig. }
% \renewcommand{\tablename}{Table }


\title{タイトルタイトル}
\author{なまえなまえ}
\date{\today}

\hypersetup{
    pdfauthor={川田侑弥},
    pdftitle={title},
    pdfsubject={},
    pdfkeywords={report;},
    pdfproducer={LaTeX},
    pdfcreator={pdfLaTeX},
    pdfduplex={none},     %Alt.: Simplex or DuplexFlipShortEdge 
    pdflang={jp}, % en
    bookmarksnumbered=true,
    colorlinks=false,
    pdfborder={0 0 0},
}

\begin{document}
%=====表紙=====%
% \begin{titlepage}
    \begin{center}
        \vspace*{40truept}
        {\huge \textbf{教科の名前など}} \\ % タイトル
        \vspace{60truept}
        {\Large 課題名\quad タイトル} \\
        
        \vfill

        \begin{tabular}{ll}
            報告者 \hspace{20mm}   & 創造工学科\ 5年4組9番\quad 川田\ 侑弥  \\
            報告書提出日           & 令和x年xx月xx日(x) \\
            % 再提出日           & 令和x年xx月xx日(x)
        \end{tabular}
    \end{center}
\end{titlepage}
%=====表紙終わり=====%

\newpage
                     % 課題用の表紙
% \begin{titlepage}
    \begin{center}
        \vspace*{40truept}
        {   
            \Large \textbf{電気電子工学実験Ⅲ報告書}
        } \\ % タイトル
        
        \vspace{30truept}
        {
            \large 実験題目\quad テーマテーマテーマ
        } \\
        \vspace{60truept}
        { 実験日\quad 令和x年xx月xx日(x)} \\
        \vspace{10truept}
        {   
            天候 \ 晴れ
            % 天候 \ 曇り
            % 天候 \ 雨
            % 天候 \ 雪
            \hspace{6truept} 室温 \ 42 $^\circ \rm{C}$
            \hspace{6truept} 湿度 \ 64 \%
        } \\

        \vspace{280truept}

        \begin{tabular}{>{\centering}p{5cm} p{30mm} p{70mm}}
            実験班   & 99班 & ~\\
            報告者   & \multicolumn{2}{l}{創造工学科\ 5年4組xx番 川田~侑弥} \\
            共同実験者   & 高専~太郎 & 高専~花子  \\
                        & カーター~ユーヤ & ジョン~スミス \\
                        % & 予備~太郎 & 予備~花子  \\
                        % & 予備~太郎 & 予備~花子  \\
        \end{tabular}

        \vfill

        \begin{tabular}{p{30mm} p{40mm}}
            報告書提出日 & 令和x年xx月xx日(x) \\
            再提出日   & 令和x年xx月xx日(x)
        \end{tabular}
    \end{center}
\end{titlepage}
%=====表紙終わり=====%

\newpage      % 実験レポート用の表紙
\makeatletter
\begin{flushright}
    \@date
\end{flushright}
\makeatother

\begin{flushleft}
    あああああ
\end{flushleft}

\vspace{10truept}

\makeatletter
\begin{center}
    {\huge \textbf{\@title}}
\end{center}

\makeatother

\vspace{30truept}                           % 共有資料用の表紙
% \maketitle

%=====本文=====%
\section{aaa}




%=====参考文献=====%
% 参考文献を手動で書く場合(以下5行のコメントアウト解除)
% \vspace{10mm}
% \begin{thebibliography}{9}
%     \bibitem{sasaki2013} 佐々木太郎, \quad フィードバック制御入門, \quad 株式会社xxx社 ,\quad 2013年
%     \bibitem{jiteisu2024} "時定数", ウィキペディア, \quad \url{https://ja.wikipedia.org/wiki/%E6%99%82%E5%AE%9A%E6%95%B0}, \quad (参照 2024-06-09)
% \end{thebibliography}


%参考文献の取り込み(library.bibというファイルが存在する場合)
% BibTexを使う場合
% \vspace{10mm}
% \bibliography{library}
% \bibliographystyle{junsrt}      % 参考文献のフォーマット

\end{document}

%%%%%%%%%%%%%%%%%%%%%%%%%%%%%%

% You can make it. :)